\documentclass[
	fontsize=12pt,           % Leitlinien sprechen von Schriftgröße 12.
	paper=A4,
	twoside=false,
	listof=totoc,            % Tabellen- und Abbildungsverzeichnis ins Inhaltsverzeichnis
	bibliography=totoc,
	headsepline]{scrreprt}
\usepackage[onehalfspacing]{setspace}
\usepackage[utf8]{inputenc}
\usepackage[printonlyused]{acronym}
\usepackage[german]{babel}
\usepackage[T1]{fontenc}
\usepackage{url}
\usepackage{amsmath}
\usepackage{amsfonts}
\usepackage{amssymb} 
\usepackage{listings}
\usepackage{graphicx}
\usepackage{wrapfig, framed}
\usepackage{float}
\usepackage{multicol}
\usepackage{hyperref}
\usepackage[a4paper,left=3cm, right=3cm, top=4cm, bottom=4cm]{geometry}
\usepackage[dvipsnames, svgnames]{xcolor}
\usepackage{scrhack}
\usepackage{prettyref}
\usepackage{mathrsfs,amssymb}
\usepackage{todonotes}
\usepackage{forest}
\usepackage[xindy,toc,automake]{glossaries}
\usepackage{xr}
\usepackage[
	backend = biber,                % Verweis auf biber
	language = auto,
	style = numeric,                % Nummerierung der Quellen mit Zahlen
	sorting = none,                 % none = Sortierung nach der Erscheinung im Dokument
	sortcites = true,               % Sortiert die Quellen innerhalb eines cite-Befehls
	block = space,                  % Extra Leerzeichen zwischen Blocks
	hyperref = true,                % Links sind klickbar auch in der Quelle
	%backref = true,                % Referenz, auf den Text an die zitierte Stelle
	bibencoding = auto,
	giveninits = true,              % Vornamen werden abgekürzt
	doi=false,                      % DOI nicht anzeigen
	isbn=false,                     % ISBN nicht anzeigen
    alldates=short                  % Datum immer als DD.MM.YYYY anzeigen
]{biblatex}
\DeclareNameAlias{default}{last-first}  % Nachname vor dem Vornamen
\AtBeginBibliography{\renewcommand{\multinamedelim}{\addslash\space
}\renewcommand{\finalnamedelim}{\multinamedelim}}  % Schrägstrich zwischen den Autorennamen
\DefineBibliographyStrings{german}{
  urlseen = {Einsichtnahme:},                      % Ändern des Titels von "besucht am"
}
\usepackage[autostyle=true,german=quotes]{csquotes}
\addbibresource{literatur.bib}

%Paragraph abstand 
\setlength{\parindent}{0em}
\setlength{\parskip}{1em}

% ---- Abstand verkleinern von der Überschrift 
\renewcommand*{\chapterheadstartvskip}{\vspace*{.4\baselineskip}}

%Commands für das Deckblatt
\newcommand{\Autor}{Karl Wolf}
\newcommand{\MatrikelNummer}{8}
\newcommand{\Was}{Projektbericht}
\newcommand{\Titel}{Optimierung von Programmen}
\newcommand{\AbgabeDatum}{11 Januar 2021}
\newcommand{\Dauer}{Dauer}
\newcommand{\Abschluss}{Master of Science}
\newcommand{\Studiengang}{Informatik}

%Commands für Codebeispiele
\newcommand{\code}[1]{\texttt{#1}}
\newcommand{\ccode}[2]{\textcolor{#1}{#2}}
\newcommand{\mquote}[1]{\grqq {#1}\grqq}

\lstset{
    tabsize=2, % tab space width
    columns=fullflexible,
    xleftmargin=2em,
    showstringspaces=false, % don't mark spaces in strings
    numbers=left, % display line numbers on the left
    commentstyle=\color{green}, % comment color
    keywordstyle=\color{blue}, % keyword color
    stringstyle=\color{red}, % string color
    basicstyle=\rmfamily\small,
    captionpos=b, 
    escapeinside={<@}{@>},
    numberstyle=\tiny,
}

\renewcommand{\lstlistingname}{Codebeispiel}
\let\origthelstnumber\thelstnumber
\newcommand*\Suppressnumber{%
  \lst@AddToHook{OnNewLine}{%
    \let\thelstnumber\relax%
     \advance\c@lstnumber-\@ne\relax%
    }%
}

\newcommand*\Reactivatenumber[1]{%
  \setcounter{lstnumber}{\numexpr#1-1\relax}
  \lst@AddToHook{OnNewLine}{%
   \let\thelstnumber\origthelstnumber%
   \refstepcounter{lstnumber}
  }%
}


\lstdefinelanguage{JavaScript}{
  keywords = [1]{class, export, boolean, throw, implements, import, this, function, test, switch, case, extends, var, telemetry, null, new},
  keywords = [2]{return, const, if, throw, else, import, from, try, await, default, render},
  keywords = [3]{fetch, dispatch, failed, fulfilled, then, get, checkVersion, json, catch, min, apply, max, map, keys, groupBy, isValidFunction, values, includes, reduce, concat, sort, getTime, splice},
  keywordstyle = [1]{\color{blue}},
  keywordstyle = [2]{\color{violet}},
  keywordstyle = [3]{\color{orange}},
  sensitive=false,
  comment=[l]{//},
  morecomment=[s]{/*}{*/},
  commentstyle=\color{green}\ttfamily,
  stringstyle=\color{brown}\ttfamily,
  morestring=[b]',
  morestring=[b]"
}

\lstdefinelanguage{xBase}{
  keywords = [1]{function, local, next, return, for, to, endif},
  keywords = [2]{AADD, ExportComponentFunction},
  keywords = [3]{DEFAULT_STATE, PRODUCTION_BAR_SELECTED, ACTION_TYPE},
  keywordstyle = [1]{\color{blue}},
  keywordstyle = [2]{\color{CornflowerBlue}},
  keywordstyle = [3]{\color{orange}},
  sensitive=false,
  comment=[l]{//},
  morecomment=[s]{/*}{*/},
  commentstyle=\textcolor{DarkGreen}\ttfamily,
  stringstyle=\color{brown}\ttfamily,
  morestring=[b]',
  morestring=[b]",
  literate={:=}{{\red{$:=$}}}2
}
\newcommand{\red}[1]{\textcolor{red}{#1}}
\newcommand{\green}[1]{\textcolor{DarkGreen}{#1}}
\newcommand{\blue}[1]{\textcolor{DarkBlue}{#1}}
\lstdefinelanguage{sharpC}{
  keywords = [1]{string, var, await, return, for, public, static, new, if, null, private},
  keywords = [2]{String, JObject, JArray, List, CycleLine, FunctionType, CycleType, Component, Comparison},
  keywordstyle = [1]{\color{blue}},
  keywordstyle = [2]{\color{DarkGreen}},
  sensitive=false,
  comment=[l]{//},
  morecomment=[s]{/*}{*/},
  commentstyle=\color{green}\ttfamily,
  stringstyle=\color{brown}\ttfamily,
  morestring=[b]',
  morestring=[b]",
  literate={:=}{{\CodeSymbol{$:=$}}}2
}

\lstdefinelanguage{JSON}{
  keywords = [1]{string, var, await, return, for, public, static, new, if, null},
  keywords = [2]{String, JObject, JArray, List, CycleLine, FunctionType, Comparison},
  keywordstyle = [1]{\color{blue}},
  keywordstyle = [2]{\color{DarkGreen}},
  sensitive=false,
  comment=[l]{//},
  commentstyle=\color{green}\ttfamily,
}

